% Generated by Sphinx.
\def\sphinxdocclass{report}
\documentclass[letterpaper,10pt,english]{sphinxmanual}
\usepackage[utf8]{inputenc}
\DeclareUnicodeCharacter{00A0}{\nobreakspace}
\usepackage{cmap}
\usepackage[T1]{fontenc}
\usepackage{babel}
\usepackage{times}
\usepackage[Bjarne]{fncychap}
\usepackage{longtable}
\usepackage{sphinx}
\usepackage{multirow}


\title{PYQCTools Documentation}
\date{August 19, 2016}
\release{1.0}
\author{Enrico Ronca}
\newcommand{\sphinxlogo}{}
\renewcommand{\releasename}{Release}
\makeindex

\makeatletter
\def\PYG@reset{\let\PYG@it=\relax \let\PYG@bf=\relax%
    \let\PYG@ul=\relax \let\PYG@tc=\relax%
    \let\PYG@bc=\relax \let\PYG@ff=\relax}
\def\PYG@tok#1{\csname PYG@tok@#1\endcsname}
\def\PYG@toks#1+{\ifx\relax#1\empty\else%
    \PYG@tok{#1}\expandafter\PYG@toks\fi}
\def\PYG@do#1{\PYG@bc{\PYG@tc{\PYG@ul{%
    \PYG@it{\PYG@bf{\PYG@ff{#1}}}}}}}
\def\PYG#1#2{\PYG@reset\PYG@toks#1+\relax+\PYG@do{#2}}

\expandafter\def\csname PYG@tok@gd\endcsname{\def\PYG@tc##1{\textcolor[rgb]{0.63,0.00,0.00}{##1}}}
\expandafter\def\csname PYG@tok@gu\endcsname{\let\PYG@bf=\textbf\def\PYG@tc##1{\textcolor[rgb]{0.50,0.00,0.50}{##1}}}
\expandafter\def\csname PYG@tok@gt\endcsname{\def\PYG@tc##1{\textcolor[rgb]{0.00,0.27,0.87}{##1}}}
\expandafter\def\csname PYG@tok@gs\endcsname{\let\PYG@bf=\textbf}
\expandafter\def\csname PYG@tok@gr\endcsname{\def\PYG@tc##1{\textcolor[rgb]{1.00,0.00,0.00}{##1}}}
\expandafter\def\csname PYG@tok@cm\endcsname{\let\PYG@it=\textit\def\PYG@tc##1{\textcolor[rgb]{0.25,0.50,0.56}{##1}}}
\expandafter\def\csname PYG@tok@vg\endcsname{\def\PYG@tc##1{\textcolor[rgb]{0.73,0.38,0.84}{##1}}}
\expandafter\def\csname PYG@tok@m\endcsname{\def\PYG@tc##1{\textcolor[rgb]{0.13,0.50,0.31}{##1}}}
\expandafter\def\csname PYG@tok@mh\endcsname{\def\PYG@tc##1{\textcolor[rgb]{0.13,0.50,0.31}{##1}}}
\expandafter\def\csname PYG@tok@cs\endcsname{\def\PYG@tc##1{\textcolor[rgb]{0.25,0.50,0.56}{##1}}\def\PYG@bc##1{\setlength{\fboxsep}{0pt}\colorbox[rgb]{1.00,0.94,0.94}{\strut ##1}}}
\expandafter\def\csname PYG@tok@ge\endcsname{\let\PYG@it=\textit}
\expandafter\def\csname PYG@tok@vc\endcsname{\def\PYG@tc##1{\textcolor[rgb]{0.73,0.38,0.84}{##1}}}
\expandafter\def\csname PYG@tok@il\endcsname{\def\PYG@tc##1{\textcolor[rgb]{0.13,0.50,0.31}{##1}}}
\expandafter\def\csname PYG@tok@go\endcsname{\def\PYG@tc##1{\textcolor[rgb]{0.20,0.20,0.20}{##1}}}
\expandafter\def\csname PYG@tok@cp\endcsname{\def\PYG@tc##1{\textcolor[rgb]{0.00,0.44,0.13}{##1}}}
\expandafter\def\csname PYG@tok@gi\endcsname{\def\PYG@tc##1{\textcolor[rgb]{0.00,0.63,0.00}{##1}}}
\expandafter\def\csname PYG@tok@gh\endcsname{\let\PYG@bf=\textbf\def\PYG@tc##1{\textcolor[rgb]{0.00,0.00,0.50}{##1}}}
\expandafter\def\csname PYG@tok@ni\endcsname{\let\PYG@bf=\textbf\def\PYG@tc##1{\textcolor[rgb]{0.84,0.33,0.22}{##1}}}
\expandafter\def\csname PYG@tok@nl\endcsname{\let\PYG@bf=\textbf\def\PYG@tc##1{\textcolor[rgb]{0.00,0.13,0.44}{##1}}}
\expandafter\def\csname PYG@tok@nn\endcsname{\let\PYG@bf=\textbf\def\PYG@tc##1{\textcolor[rgb]{0.05,0.52,0.71}{##1}}}
\expandafter\def\csname PYG@tok@no\endcsname{\def\PYG@tc##1{\textcolor[rgb]{0.38,0.68,0.84}{##1}}}
\expandafter\def\csname PYG@tok@na\endcsname{\def\PYG@tc##1{\textcolor[rgb]{0.25,0.44,0.63}{##1}}}
\expandafter\def\csname PYG@tok@nb\endcsname{\def\PYG@tc##1{\textcolor[rgb]{0.00,0.44,0.13}{##1}}}
\expandafter\def\csname PYG@tok@nc\endcsname{\let\PYG@bf=\textbf\def\PYG@tc##1{\textcolor[rgb]{0.05,0.52,0.71}{##1}}}
\expandafter\def\csname PYG@tok@nd\endcsname{\let\PYG@bf=\textbf\def\PYG@tc##1{\textcolor[rgb]{0.33,0.33,0.33}{##1}}}
\expandafter\def\csname PYG@tok@ne\endcsname{\def\PYG@tc##1{\textcolor[rgb]{0.00,0.44,0.13}{##1}}}
\expandafter\def\csname PYG@tok@nf\endcsname{\def\PYG@tc##1{\textcolor[rgb]{0.02,0.16,0.49}{##1}}}
\expandafter\def\csname PYG@tok@si\endcsname{\let\PYG@it=\textit\def\PYG@tc##1{\textcolor[rgb]{0.44,0.63,0.82}{##1}}}
\expandafter\def\csname PYG@tok@s2\endcsname{\def\PYG@tc##1{\textcolor[rgb]{0.25,0.44,0.63}{##1}}}
\expandafter\def\csname PYG@tok@vi\endcsname{\def\PYG@tc##1{\textcolor[rgb]{0.73,0.38,0.84}{##1}}}
\expandafter\def\csname PYG@tok@nt\endcsname{\let\PYG@bf=\textbf\def\PYG@tc##1{\textcolor[rgb]{0.02,0.16,0.45}{##1}}}
\expandafter\def\csname PYG@tok@nv\endcsname{\def\PYG@tc##1{\textcolor[rgb]{0.73,0.38,0.84}{##1}}}
\expandafter\def\csname PYG@tok@s1\endcsname{\def\PYG@tc##1{\textcolor[rgb]{0.25,0.44,0.63}{##1}}}
\expandafter\def\csname PYG@tok@gp\endcsname{\let\PYG@bf=\textbf\def\PYG@tc##1{\textcolor[rgb]{0.78,0.36,0.04}{##1}}}
\expandafter\def\csname PYG@tok@sh\endcsname{\def\PYG@tc##1{\textcolor[rgb]{0.25,0.44,0.63}{##1}}}
\expandafter\def\csname PYG@tok@ow\endcsname{\let\PYG@bf=\textbf\def\PYG@tc##1{\textcolor[rgb]{0.00,0.44,0.13}{##1}}}
\expandafter\def\csname PYG@tok@sx\endcsname{\def\PYG@tc##1{\textcolor[rgb]{0.78,0.36,0.04}{##1}}}
\expandafter\def\csname PYG@tok@bp\endcsname{\def\PYG@tc##1{\textcolor[rgb]{0.00,0.44,0.13}{##1}}}
\expandafter\def\csname PYG@tok@c1\endcsname{\let\PYG@it=\textit\def\PYG@tc##1{\textcolor[rgb]{0.25,0.50,0.56}{##1}}}
\expandafter\def\csname PYG@tok@kc\endcsname{\let\PYG@bf=\textbf\def\PYG@tc##1{\textcolor[rgb]{0.00,0.44,0.13}{##1}}}
\expandafter\def\csname PYG@tok@c\endcsname{\let\PYG@it=\textit\def\PYG@tc##1{\textcolor[rgb]{0.25,0.50,0.56}{##1}}}
\expandafter\def\csname PYG@tok@mf\endcsname{\def\PYG@tc##1{\textcolor[rgb]{0.13,0.50,0.31}{##1}}}
\expandafter\def\csname PYG@tok@err\endcsname{\def\PYG@bc##1{\setlength{\fboxsep}{0pt}\fcolorbox[rgb]{1.00,0.00,0.00}{1,1,1}{\strut ##1}}}
\expandafter\def\csname PYG@tok@kd\endcsname{\let\PYG@bf=\textbf\def\PYG@tc##1{\textcolor[rgb]{0.00,0.44,0.13}{##1}}}
\expandafter\def\csname PYG@tok@ss\endcsname{\def\PYG@tc##1{\textcolor[rgb]{0.32,0.47,0.09}{##1}}}
\expandafter\def\csname PYG@tok@sr\endcsname{\def\PYG@tc##1{\textcolor[rgb]{0.14,0.33,0.53}{##1}}}
\expandafter\def\csname PYG@tok@mo\endcsname{\def\PYG@tc##1{\textcolor[rgb]{0.13,0.50,0.31}{##1}}}
\expandafter\def\csname PYG@tok@mi\endcsname{\def\PYG@tc##1{\textcolor[rgb]{0.13,0.50,0.31}{##1}}}
\expandafter\def\csname PYG@tok@kn\endcsname{\let\PYG@bf=\textbf\def\PYG@tc##1{\textcolor[rgb]{0.00,0.44,0.13}{##1}}}
\expandafter\def\csname PYG@tok@o\endcsname{\def\PYG@tc##1{\textcolor[rgb]{0.40,0.40,0.40}{##1}}}
\expandafter\def\csname PYG@tok@kr\endcsname{\let\PYG@bf=\textbf\def\PYG@tc##1{\textcolor[rgb]{0.00,0.44,0.13}{##1}}}
\expandafter\def\csname PYG@tok@s\endcsname{\def\PYG@tc##1{\textcolor[rgb]{0.25,0.44,0.63}{##1}}}
\expandafter\def\csname PYG@tok@kp\endcsname{\def\PYG@tc##1{\textcolor[rgb]{0.00,0.44,0.13}{##1}}}
\expandafter\def\csname PYG@tok@w\endcsname{\def\PYG@tc##1{\textcolor[rgb]{0.73,0.73,0.73}{##1}}}
\expandafter\def\csname PYG@tok@kt\endcsname{\def\PYG@tc##1{\textcolor[rgb]{0.56,0.13,0.00}{##1}}}
\expandafter\def\csname PYG@tok@sc\endcsname{\def\PYG@tc##1{\textcolor[rgb]{0.25,0.44,0.63}{##1}}}
\expandafter\def\csname PYG@tok@sb\endcsname{\def\PYG@tc##1{\textcolor[rgb]{0.25,0.44,0.63}{##1}}}
\expandafter\def\csname PYG@tok@k\endcsname{\let\PYG@bf=\textbf\def\PYG@tc##1{\textcolor[rgb]{0.00,0.44,0.13}{##1}}}
\expandafter\def\csname PYG@tok@se\endcsname{\let\PYG@bf=\textbf\def\PYG@tc##1{\textcolor[rgb]{0.25,0.44,0.63}{##1}}}
\expandafter\def\csname PYG@tok@sd\endcsname{\let\PYG@it=\textit\def\PYG@tc##1{\textcolor[rgb]{0.25,0.44,0.63}{##1}}}

\def\PYGZbs{\char`\\}
\def\PYGZus{\char`\_}
\def\PYGZob{\char`\{}
\def\PYGZcb{\char`\}}
\def\PYGZca{\char`\^}
\def\PYGZam{\char`\&}
\def\PYGZlt{\char`\<}
\def\PYGZgt{\char`\>}
\def\PYGZsh{\char`\#}
\def\PYGZpc{\char`\%}
\def\PYGZdl{\char`\$}
\def\PYGZhy{\char`\-}
\def\PYGZsq{\char`\'}
\def\PYGZdq{\char`\"}
\def\PYGZti{\char`\~}
% for compatibility with earlier versions
\def\PYGZat{@}
\def\PYGZlb{[}
\def\PYGZrb{]}
\makeatother

\renewcommand\PYGZsq{\textquotesingle}

\begin{document}

\maketitle
\tableofcontents
\phantomsection\label{index::doc}


PYQCTools is a collection of python scripts useful to dump quantum chemistry integrals and to
perform data post-processing for different quantum chemistry methods.

PYQCTools requires the following prerequisites to work:
\begin{itemize}
\item {} 
Python 2.6, 2.7, 3.2, 3.3, 3.4

\item {} 
Numpy 1.6.2 or higher

\item {} 
Scipy 0.10 or higher (0.12.0 or higher for python 3.3, 3.4)

\item {} 
\href{https://github.com/sunqm/pyscf}{PySCF} for Integrals Calculation in Integrals Dumpings scripts.

\end{itemize}


\chapter{Contents}
\label{index:contents}\label{index:welcome-to-pyqctools-s-documentation}

\section{Tools for Omega Space Green's Functions}
\label{omegagf:tools-for-omega-space-green-s-functions}\label{omegagf::doc}\begin{description}
\item[{\code{gf\_trace.py}: Calculate the Density of States (DOS) value from an \(\omega\)-dependent Green's Function.}] \leavevmode
It makes the trace of the Green's Function associated with a certain frequency value.

\textbf{Example}:

\begin{Verbatim}[commandchars=\\\{\}]
\PYG{k+kn}{from} \PYG{n+nn}{PYQCTools.Omega\PYGZus{}GF} \PYG{k+kn}{import} \PYG{n}{gf\PYGZus{}trace}

\PYG{n}{gf\PYGZus{}trace}\PYG{o}{.}\PYG{n}{run}\PYG{p}{(}\PYG{n}{green}\PYG{o}{.}\PYG{n}{txt}\PYG{p}{,} \PYG{n}{omega\PYGZus{}value}\PYG{p}{)}
\end{Verbatim}

\code{green.txt}: formatted text file containing the Green's function, \code{omega\_value}: double frequency value.

\end{description}


\section{Tools for Real-Time Green's Functions}
\label{rtgf::doc}\label{rtgf:tools-for-real-time-green-s-functions}\begin{description}
\item[{\code{rtgf.py}: Calculate the Density of States (DOS) values during a time propagation.}] \leavevmode
It makes the trace of time-dependent Green's Functions calculated
along a time propagation and return the DOS values as a function of time both for the real and for the imaginary part of the Green's Function.
\begin{quote}

\textbf{Example}:

\begin{Verbatim}[commandchars=\\\{\}]
\PYG{k+kn}{from} \PYG{n+nn}{PYQCTools.RT\PYGZus{}GF} \PYG{k+kn}{import} \PYG{n}{rtgf}

\PYG{n}{rtgf}\PYG{o}{.}\PYG{n}{run}\PYG{p}{(}\PYG{n}{prop\PYGZus{}time}\PYG{p}{,} \PYG{n}{time\PYGZus{}step}\PYG{p}{,} \PYG{n}{scratch}\PYG{p}{)}
\end{Verbatim}

\code{prop\_time}: double value of the full propagation time (period), \code{time\_step}: double value of the time-step,
\code{scratch}: directory containing text files of the real (green.\$t.\$t.txt) and imaginary (green.30000+\$t.30000+\$t.txt) Green's Functions, where \$t indicate the specific time-step.
\end{quote}

The script save two output files, \code{rt\_real.txt} and \code{rt\_imag.txt}, containing the real and imaginary part of the time-dependent DOS respectively.

\item[{\code{fft.py}:  Perform the fourier transform of the time-dependent Density of States (DOS).}] \leavevmode
It reads the \code{rt\_real.txt} and \code{rt\_imag.txt} generated by the \code{rtgf.py} script and produces the \code{ldos.out} and \code{real\_part.txt} files containing the imaginary and real parts of the \emph{omega}-dependent DOS respectively.
\begin{quote}

\textbf{Example}:

\begin{Verbatim}[commandchars=\\\{\}]
\PYG{k+kn}{from} \PYG{n+nn}{PYQCTools.RT\PYGZus{}GF} \PYG{k+kn}{import} \PYG{n}{fft}

\PYG{n}{fft}\PYG{o}{.}\PYG{n}{run}\PYG{p}{(}\PYG{n}{broad}\PYG{p}{,} \PYG{n}{rem\PYGZus{}add}\PYG{p}{)}
\end{Verbatim}

\code{broad}: double value of the imaginary broadening,
\code{rem\_add}: string specifying if we are working with the addition or removal part of the Green's Function. It can assume only the values `add' or `rem'.
\end{quote}

\item[{\code{rt1rdm\_builder.py}: Build the Real and Imaginary parts of a Real-Time 1RDM at every time step starting from DMRG data calculated by the Block Code.}] \leavevmode
It opportunely combines the 1RDM components read from files \code{onepdm.\$t.\$t.txt} where \code{\$t} is of the order of 1,2,3,..., 100001,10002,100003,... and 200001,200002,200003,... for the Real-Real, Imag-Real and Imag-Imag respectively.
\begin{quote}

\textbf{Example}:

\begin{Verbatim}[commandchars=\\\{\}]
\PYG{k+kn}{from} \PYG{n+nn}{PYQCTools.RT\PYGZus{}GF} \PYG{k+kn}{import} \PYG{n}{rt1rdm\PYGZus{}builder}

\PYG{n}{rt1rdm\PYGZus{}builder}\PYG{o}{.}\PYG{n}{run}\PYG{p}{(}\PYG{n}{prop\PYGZus{}time}\PYG{p}{,} \PYG{n}{time\PYGZus{}step}\PYG{p}{,} \PYG{n}{scratch}\PYG{p}{)}
\end{Verbatim}

\code{prop\_time}: double value of the full propagation time (period), \code{time\_step}: double value of the time-step,
\code{scratch}: directory containing text files of the 1RDM components at every time-step.
\end{quote}

\item[{\code{extrapolation.py}: Perform linear prediction to extend the total propagation time of a time propagation.}] \leavevmode\begin{quote}
\begin{quote}

It reads N points of time-dependent DOS inside the files \code{rt\_real.txt} and \code{rt\_imag.txt} and use the last N/2 data to predict the following N points.
\end{quote}

\textbf{Example}:

\begin{Verbatim}[commandchars=\\\{\}]
\PYG{k+kn}{from} \PYG{n+nn}{PYQCTools.RT\PYGZus{}GF} \PYG{k+kn}{import} \PYG{n}{extrapolation}

\PYG{n}{extrapolation}\PYG{o}{.}\PYG{n}{run}\PYG{p}{(}\PYG{n}{full\PYGZus{}range}\PYG{p}{)}
\end{Verbatim}

\code{full\_range}: boolean variable. If it is true is return the full range of calculated and predicted values, if it is false it returns only the predicted values.
\end{quote}

The script produces \code{new\_full\_data.out} files if full\_range = true otherwise it produces \code{predicted.out} output files.

\item[{\code{iter\_extrapolation.py}: Perform an interative linear prediction to extend the total propagation time of a time propagation.}] \leavevmode\begin{quote}
\begin{quote}

It reads 4 points of the time-dependent DOS inside the files \code{rt\_real.txt} and \code{rt\_imag.txt} and use the last 2 of them to predict the following N points.
\end{quote}

\textbf{Example}:

\begin{Verbatim}[commandchars=\\\{\}]
\PYG{k+kn}{from} \PYG{n+nn}{PYQCTools.RT\PYGZus{}GF} \PYG{k+kn}{import} \PYG{n}{iter\PYGZus{}extrapolation}

\PYG{n}{iter\PYGZus{}extrapolation}\PYG{o}{.}\PYG{n}{run}\PYG{p}{(}\PYG{n}{N}\PYG{p}{)}
\end{Verbatim}

\code{N}: integer variable specifying the total number of points that need to be predicted.
\end{quote}

The script produces \code{new\_full\_real.out} and \code{new\_full\_imag.out} output files with the real and imaginary parts of the exteded time-dependent DOS respectively.

\end{description}


\section{Tools for Integrals Dumping}
\label{integralsdump:tools-for-integrals-dumping}\label{integralsdump::doc}\begin{description}
\item[{\code{Integrals\_dump.py}: It dumps 1 and 2-elctron integrals in the MO basis inside a CASCI space in FCIDUMP format.}] \leavevmode
The PySCF input to calculate the integrals is already included in the script.

\item[{\code{DipoleIntegrals\_dump.py}: It dumps dipole integrals in the MO basis in a CASCI space in FCIDUMP format.}] \leavevmode
The PySCF input to calculate the integrals is already included in the script.

\item[{\code{LowdinOrtho\_Integrals.py}: It dumps 1 and 2-electron integrals in the Localized basis obtained by Lowdin Orthogonalization in FCIDUMP format.}] \leavevmode
The PySCF input to calculate the integrals is already included in the script.

\end{description}

\code{hubbard\_1d}: It dumps 1 and 2-electron integrals got the 1D Hubbard model.
\begin{quote}

\textbf{Example}:

\begin{Verbatim}[commandchars=\\\{\}]
\PYG{k+kn}{from} \PYG{n+nn}{PYQCTools.Integrals\PYGZus{}dump} \PYG{k+kn}{import} \PYG{n}{hubbard\PYGZus{}1d}

\PYG{n}{hubbard\PYGZus{}1d}\PYG{o}{.}\PYG{n}{run}\PYG{p}{(}\PYG{n}{nsites}\PYG{p}{,} \PYG{n}{t}\PYG{p}{,} \PYG{n}{U}\PYG{p}{,} \PYG{n}{output}\PYG{p}{,} \PYG{n}{pbc}\PYG{p}{)}
\end{Verbatim}

\code{nsites}: Number of sites, \code{t}: Hopping constant, \code{U}: Coupling constant, \code{output}: Output file name, \code{pbc}: `True' of `False' respectively if periodic boudary conditions need to be included or not.
\end{quote}
\begin{description}
\item[{\code{MPSPT\_integrals.py}: It dumps the DYALL and PERTURB files to run NEVPT2 calculations by MPS-PT using the BLOCK Code for DMRG.}] \leavevmode
Integrals are dumped in FCIDUMP format.
The PySCF input to calculate the integrals is already included in the script.

\end{description}

You can also download the \href{https://raw.github.com/eronca/PYQCTools/gh-pages/latex/PYQCTools.pdf}{PDF version} of this manual.



\renewcommand{\indexname}{Index}
\printindex
\end{document}
